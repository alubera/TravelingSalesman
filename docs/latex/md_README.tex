\subsection*{Description}

This is a project for A\-C\-M\-S 60212 -\/ Advanced Scientic Computing -\/ by John Huber, Andrew Lubera, and Andrew Good. Our project seeks to implement the Traveling Salesman Problem (T\-S\-P) in C++ and broadly explore parallel algorithms and heuristics that can be used to best approximate the solution to the problem.

\subsection*{Building}

C\-Make was used to build and link the necessary Boost libraries.

\subsubsection*{C\-R\-C}

If you are on the C\-R\-C, C\-Make 3.\-6.\-3 is already installed. Thus, to build, do the following in the root directory\-: ``` mkdir build cd build cmake .. make ``` \subsubsection*{O\-S\-X}

If you are compiling on a local machine using O\-S\-X, it will be necessary for you to install C\-Make first. It is easiest to use homebrew to install C\-Make. Do the following\-: ``` brew install cmake ``` \subsubsection*{Windows}

If you are using a windows machine, you will need to visit \href{https://cmake.org/download/}{\tt https\-://cmake.\-org/download/} where there are precompiled binaries for download.

\subsection*{Running}

After you build the project, you will still be in the build/ directory. To run the executable, do the following\-: ``` cd bin ./main.out ``` \subsection*{Testing}

We have included the option to run Boost unit tests on our code. We believe that this prevents careless mistakes and increases reproducibility. The test.\-out executable is automatically build. If you are in the build/ directory, to test, do the following\-: ``` cd bin ./test.out ``` \subsection*{Data}

The data that we use in this project comes from \href{https://gist.github.com/Miserlou/c5cd8364bf9b2420bb29}{\tt https\-://gist.\-github.\-com/\-Miserlou/c5cd8364bf9b2420bb29} and is in J\-S\-O\-N format. It contains information, including latitude and longitude coordinates, for the 1,000 most populous cities in the continental United States. 